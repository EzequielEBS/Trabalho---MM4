\maketitle

\section{Introdução}

O Brasil é o maior produtor de frutas cítricas e cana-de-açúcar do mundo, sendo responsável por produzir cerca de 20 milhões de toneladas de citros e mais de 600 milhões de toneladas de cana-de-açúcar nos últimos anos. No entanto, essas plantações são atacadas por algumas pragas, gerando sérios prejuízos à produção. As primeiras são atingidas pela morte súbita dos citros (PEIXOTO,
BARROS, BASSANEZI, 2005) \cite{mp_lb_rb_2005}, doença que chegou ao extremo de provocar a morte de grandes plantações no estado de São Paulo. Já a cana-de-açúcar é afetada pela \textit{Diatraea saccharalis} (JESUS, LIMA, 2017) \cite{ij_al_2017}, que gera inúmeras falhas na germinação, morte das gemas e perda de peso, diminuindo a pureza.

No caso da cana-de-açúcar, a \textit{Diatraea saccharalis}, conhecida como broca da cana-de-açúcar, passa a maior parte do tempo dentro da cana, dificultando a ação de agentes químicos. Como consequência, o controle biológico tem sido a forma mais eficiente de combate, com a utilização de outros insetos predadores presentes no canavial. Já nos citros, pesquisadores acreditam que a doença é causada por um vírus transmitido por insetos conhecidos como pulgões, os quais possuem as joaninhas como predadores naturais mais conhecidos.

Nesse sentido, a Modelagem Matemática é uma ferramenta importante para encontrar maneiras de controlar ou resolver tais problemas. Os modelos matemáticos para esse fim tiveram origem com Lotka em 1925 e Volterra em 1926, que propuseram um modelo que foi, posteriormente, denominado modelo de Lotka-Volterra, utilizado para descrever a dinâmica de sistemas do tipo predador-presa, onde uma das espécies é predadora da outra, a presa, que se alimenta de outro tipo de alimento. Esse modelo já foi utilizado para modelagem do controle biológico do controle de pragas em algumas plantações, como podemos ver no artigo de Viviane Noronha e Rosana Ferreira \cite{vn_rf_2021}, que trata do controle biológico de maneira mais geral na frutífera brasileira, e também no artigo de I. Jesus  A. D. Lima \cite{ij_al_2017}, o qual faz uma abordagem aplicada à cana-de-açúcar.

Além disso, podemos citar o modelo de Holling-Tanner que, diferentemente do modelo anterior, ``leva em consideração o efeito predação e as capacidades de suportes das populações de presas e predadores'' (SILVEIRA, GARCIA, 2020) \cite{gs_rg_2020}, proposto no artigo de Magda S. Peixoto, Laécio C. Barros e Rodney C. Bassanezi para o controle de praga nos citros \cite{mp_lb_rb_2005}. 

Em nosso trabalho, além dos modelos supracitados, vamos propor uma análise com o modelo de Rosenzweig-MacArthur, que considera fatores adicionais na população de predadores como, por exemplo, uma taxa de mortalidade \cite{id_mg_r_2021}, devido outros agentes. Em nossa modelagem, iremos admitir a existência de um superpredador sem capacidade de reprodução, o qual provoca uma taxa de mortalidade nos predadores. A seguir, vamos apresentar os modelos citados para descrever a dinâmica predador-presa, bem como utilizá-los para modelar os problemas relatados.
