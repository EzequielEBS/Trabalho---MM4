\section{Conclusão}

Analisamos três modelos com o intuito de modelar a dinâmica predador-presa nas populações de brocas e vespas e de pulgões e joaninhas. Com o modelo de Lotka-Volterra, fizemos uma representação simplificada para as primeiras populações, onde o modelo pressupõe crescimento indefinido das espécies. Em seguida, exploramos o modelo de Holling-Tanner com as populações de pulgões e joaninhas, onde pudemos contemplar uma maior representação da realidade, com o crescimento das populações limitado e dependente das interações entre elas. Por fim, tentamos modelar ambas as dinâmicas populacionais com o modelo de Rosenzweig-MacArthur, semelhante ao modelo anterior, mas que considera outras características dos predadores. Nesse modelo, percebemos um comportamento bem cíclico, onde a população de presas crescia à medida que a população de predadores diminuía.

A abordagem aqui presente se mostra de suma importância para estudar a dinâmica das populações e buscar maneiras efetivas de estabelecer um controle biológico das presas. No entanto, o grande fator limitante é a ausência de dados para a estimação dos parâmetros utilizados. Sobretudo no último modelo, devido a falta de muitos deles, tivemos que definir algumas condições de regularidade para a realização da modelagem. Ademais, esbarramos na limitação de estabilidade numérica, onde precisamos determinar intervalos de estimação dos parâmetros condizentes com a realidade e que eram estáveis numericamente.
